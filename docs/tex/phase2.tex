%! Author  = Teresa Ignatzek       % nur als Kommentar – erscheint nicht im PDF
%! Phase   = Dashboard – Phase 2   %            "

\documentclass[11pt,a4paper]{article}

% -------------------------------------------------------
%                 Pakete & Grundeinstellungen
% -------------------------------------------------------
\usepackage{xcolor}
\usepackage[utf8]{inputenc}
\usepackage[T1]{fontenc}
\usepackage[ngerman]{babel}       % deutsche Silbentrennung
\usepackage{csquotes}             % saubere Anführungszeichen
\usepackage{microtype}            % typografischer Feinschliff
\usepackage{amsmath,amssymb}      % Mathe
\usepackage{graphicx}             % Bilder einbinden
\graphicspath{{../images/}}
\usepackage{float}                % präzise Platzierung von Abbildungen
\usepackage{caption}              % Beschriftungen konfigurieren
\usepackage{subcaption}           % Teilabbildungen
\usepackage{listings}             % Code-Listings
\usepackage{hyperref}             % klickbare Links
\usepackage{cleveref}             % \cref statt Figure 1
\usepackage{geometry}             % Ränder
\usepackage{helvet}           % Arial‑ähnliche Schrift
\renewcommand{\familydefault}{\sfdefault}
\usepackage{setspace}         % Zeilenabstand
\DeclareUnicodeCharacter{202F}{\,}% schmale geschützte Leerstelle

\geometry{left=2cm,right=2cm,top=2cm,bottom=2cm}
\raggedbottom            % vermeidet überflüssige Leer­seiten

% Keine Erstzeilen‑Einrückung
\setlength{\parindent}{0pt}


% Listings-Style für Python / SQL
\lstset{
    basicstyle=\ttfamily\small,
    keywordstyle=\bfseries\color{blue!70!black},
    commentstyle=\itshape\color{gray!70},
    stringstyle=\color{green!40!black},
    frame=single,
    breaklines=true,
    tabsize=4,
    literate        = { }{{\ }}1
                      {•}{{\textbullet}}1
                      {Ä}{{\"A}}1
                      {Ö}{{\"O}}1
                      {Ü}{{\"U}}1
                      {ä}{{\"a}}1
                      {ö}{{\"o}}1
                      {ü}{{\"u}}1
                      {Ø}{{\O}}1
                      {–}{{--}}1,
}

% --- Listings: JSON definition + UTF‑8 patch ------------------------------
\lstdefinelanguage{json}{
    basicstyle=\ttfamily\small,
    keywords={true,false,null},
    keywordstyle=\color{blue!70!black},
    stringstyle=\color{green!40!black},
    commentstyle=\itshape\color{gray!70},
    numbers=left,
    numberstyle=\tiny\color{gray},
    stepnumber=1,
    numbersep=5pt,
    showstringspaces=false,
    breaklines=true,
    frame=single,
    literate=
      {Ø}{{\O}}1
      {–}{{--}}1
      {Ä}{{\"A}}1{Ö}{{\"O}}1{Ü}{{\"U}}1
      {ä}{{\"a}}1{ö}{{\"o}}1{ü}{{\"u}}1
}

% Custom style for clean Python listings
\lstdefinestyle{mypython}{
    language        = Python,
    basicstyle      = \ttfamily\footnotesize,
    keywordstyle    = \bfseries\color{blue!70!black},
    commentstyle    = \itshape\color{gray!70},
    stringstyle     = \color{green!40!black},
    frame           = single,
    breaklines      = true,
    breakindent     = 1em,
    postbreak       = \mbox{\textcolor{red}{$\hookrightarrow$}\space},
    showstringspaces= false,
    columns         = fullflexible,
    keepspaces      = true,
    literate        = {_}{{\_}}1
                      { }{{\ }}1
                      {•}{{\textbullet}}1
                      {Ä}{{\"A}}1
                      {Ö}{{\"O}}1
                      {Ü}{{\"U}}1
                      {ä}{{\"a}}1
                      {ö}{{\"o}}1
                      {ü}{{\"u}}1,
}

% -------------------------------------------------------
%                     Dokument
% -------------------------------------------------------

\begin{document}
\onehalfspacing               % 1,5‑facher Zeilenabstand


% \section entspricht Überschrift 1. Ordnung
\section{Zielsetzung und Umfang}

Dieses Projekt zielt darauf ab, ein interaktives, fahrzeugtaugliches Studien‐Dashboard („Studiennavigator“) zu entwickeln, das den individuellen Studienfortschritt visuell nachvollziehbar macht und zugleich Buchungen für künftige Module erlaubt.

In Phase 1 wurde das Grundlayout mit einem statischen SVG‐Pfad sowie ein erstes, noch unvollständiges Klassendiagramm umgesetzt.
Phase 2 erweitert dieses Fundament. Ich untersuche zunächst, wie sich die im Klassendiagramm vorgesehenen objektorientierten Konzepte (Klassen, Vererbung, Aggregation / Komposition) in Python idiomatisch mittels \texttt{@dataclass} und Typannotationen realisieren lassen und leite daraus ein verfeinertes UML‐Modell ab. Anschließend entwerfe ich eine dreischichtige Gesamtarchitektur (Controller, View, Gateway) – inklusive Login, Modulbuchung per Ortsschild‐Overlay und semesterfeiner Fortschrittsanzeige – und halte die Ergebnisse in diesem maximal fünfseitigen Reflexions‑ und Entwurfsdokument fest.


 \section{Anforderungsanalyse (Auszug)}
\subsection{Funktionale Anforderungen}
Zu den wichtigsten funktionalen Anforderungen zählen:

\begin{itemize}
  \item \textbf{Login / Authentifizierung}\\
        Ich implementiere einen einfachen, passwortgeschützten
        Anmelde‑Workflow, der die Session ID im \lstinline{Flask}‑Cookie
        speichert und nach erfolgreicher Authentifizierung den Zugriff
        auf sämtliche Dashboard‑Routen freigibt.
  \item \textbf{Ortsschild‑Overlay zur Modulbuchung}\\
        Durch Klick auf ein semestergenaues Ortsschild öffnet sich ein
        Modal‑Dialog, in dem ich die im Datenmodell definierte
        Modul­liste anzeige.  Ein \texttt{POST}‑Request ruft
        \lstinline{db_gateway.book_module()} auf und legt den Datensatz
        in \lstinline{modulbuchung} an.
  \item \textbf{Fortschrittsanzeige}\\
        Das leere \texttt{Infotainment.svg} wird per JavaScript mit dynamischen KPI‑Texten gefüllt.
        Dafür stelle ich drei Schwellenwerte in einer externen \texttt{progress.json} bereit:

\begin{lstlisting}[language=json,caption={Auszug progress.json}]
{
  "grade": {
    "fast":   "%{value} – Stabile Fahrt auf der Überholspur.",
    "medium": "%{value} – Voll im Zeitplan.",
    "slow":   "%{value} – Ich schalte einen Gang höher!"
  },
  "time": {
    "plus":  "+ %{days} Tage Puffer – Cruise Mode.",
    "minus": "- %{days} Tage – Gaspedal durchdrücken!"
  },
  "fee": {
    "open":  "%{amount} € Gebühren offen.",
    "zero":  "Alle Gebühren beglichen."
  }
}
\end{lstlisting}

        Die Python‑Logik wählt anhand von Schwellenwerten (\texttt{grade<=2.0\,→\,fast}, \texttt{days\_delta<0\,→\,minus}\,…) den passenden Satz, ersetzt Platzhalter mit Echtwerten und übergibt die Strings als \verb|{{ grade }}|, \verb|{{ time }}|, \verb|{{ fee }}| an das SVG.
        So entsteht eine sprachliche Navigation („Cruise Mode“ / „Gaspedal durchdrücken!“), die sofort den Studienstatus kommuniziert, ohne zusätzliche Farben oder Diagramme.
\end{itemize}

\subsection{Nicht-funktionale Anforderungen}
Nicht‑funktional spielen folgende Qualitätsmerkmale eine Rolle:

\begin{itemize}
  \item \textbf{Responsives Frontend}\\
        Das HTML / CSS‑Layout passt sich an die Auflösung des
        Infotainment‑Displays an und ist somit auch auf Laptops
        und Tablets nutzbar.
  \item \textbf{Persistenz}\\
        Ich verwende SQLite als leichtgewichtige, serverlose
        Datenbank, sodass das Dashboard ohne zusätzliche
        Infrastruktur lauffähig ist.
  \item \textbf{Performance}\\
        Alle Datenbankabfragen sind parametriert und kehren
        nur spaltenselektierte Result‑Sets zurück, um die
        Ladezeiten unter 200 ms zu halten.
  \item \textbf{Portabilität}\\
        Durch Containerisierung mit \texttt{docker-compose} kann das
        Projekt unter Windows, macOS und Linux identisch gestartet werden.
\end{itemize}

\section{Konzept \& UML-Modell}
\begin{figure}[htbp]
  \centering
  \includegraphics[width=.85\textwidth]{dashboard.png}
  \caption{Aktualisiertes Klassendiagramm – Phase 2}
  \label{fig:uml}
\end{figure}

\paragraph{Beziehungslogik (Stichpunkte).}
\begin{itemize}
  \item \textbf{Student \ensuremath{1\rightarrow *}\ Einschreibung}
        Aggregation – eine \emph{Einschreibung} gehört genau zu einem \emph{Student}, kann aber gelöscht werden, ohne die \emph{Student}‑Instanz zu zerstören.

  \item \textbf{Einschreibung \ensuremath{1\rightarrow 1}\ Studiengang}
        Assoziation – jede \emph{Einschreibung} verweist auf genau einen \emph{Studiengang}; beide Objekte können jedoch unabhängig voneinander existieren.

  \item \textbf{Studiengang \ensuremath{1\rightarrow *}\ StudiengangModul}
        Komposition – die Semester‑/Pflichtzuordnung (\emph{StudiengangModul}) hat keine eigene Lebensdauer außerhalb des zugehörigen \emph{Studiengangs}.

  \item \textbf{Modul \ensuremath{1\rightarrow *}\ StudiengangModul}
        Assoziation – dasselbe \emph{Modul} kann in mehreren Studiengängen auftauchen; es wird lediglich referenziert.

  \item \textbf{Einschreibung \ensuremath{1\rightarrow *}\ Modulbuchung}
        Aggregation – eine \emph{Modulbuchung} hängt logisch an einer konkreten \emph{Einschreibung}, lebt aber nach deren Löschung nicht weiter.

  \item \textbf{Modulbuchung \ensuremath{1\rightarrow 1}\ Modul}
        Assoziation – jede Buchung referenziert ein \emph{Modul}; beide Seiten bleiben unabhängig.

  \item \textbf{Student \ensuremath{1\rightarrow *}\ Gebuehr}
        Aggregation – Gebuehren werden von Studierenden ausgelöst, können aber getrennt ausgewertet werden (z.\,B. für die Buchhaltung).

  \item \textbf{Modulbuchung \ensuremath{1\rightarrow 0..1}\ Pruefungsleistung}
        Vererbung/Polymorphie – \emph{Pruefungsleistung} ist eine spezialisierte Folge einer \emph{Modulbuchung}; erst nach Abschluss entsteht die Unterklasse‐Instanz.
\end{itemize}

\section{Implementierung}
\subsection{Modul‐/Package-Struktur}
\begin{verbatim}
dashboardProject/
|-- app.py                 # Flask-Bootstrapper
|-- auth_controller.py     # Login + Session
|-- semester_controller.py # Routing /semester/<n>
|-- db_gateway.py          # CRUD-Schicht (SQLite)
|-- templates/
|   |-- index.html
|   `-- sign.html
`-- static/uploads/
    |-- Pfad.svg
    |-- Sign.svg
    `-- ...
\end{verbatim}

\subsection{Code-Ausschnitte}
\begin{lstlisting}[style=mypython,
                   caption={Ausschnitt \texttt{db\_gateway.py}},
                   label={lst:book_module}]
def book_module(student_id: int, modul_id: int, sem: int) -> None:
    sql = """
    INSERT INTO modulbuchung (student_id, modul_id, semester, status)
    VALUES (?, ?, ?, 'gebucht');
    """
    # DB_PATH = Path(__file__).with_suffix('.db')
    with sqlite3.connect(DB_PATH) as con:
        # Fremdschlüsselprüfung in SQLite einschalten
        con.execute("PRAGMA foreign_keys = ON;")
        con.execute(sql, (student_id, modul_id, sem))
\end{lstlisting}


\subsection{Datenbank‐Schema}
\begin{lstlisting}[language=SQL,caption={DDL-Extrakt}]
CREATE TABLE modulbuchung (
    id INTEGER PRIMARY KEY AUTOINCREMENT,
    student_id INTEGER NOT NULL,
    modul_id   INTEGER NOT NULL,
    semester   INTEGER,
    status     TEXT CHECK(status IN ('gebucht','bestanden')),
    note       REAL,
    FOREIGN KEY(student_id) REFERENCES student(id),
    FOREIGN KEY(modul_id)   REFERENCES modul(id)
);
\end{lstlisting}
\medskip
\textit{Hinweis:} SQLite wertet Fremdschlüssel nur aus, wenn sie Sitzung‑
spezifisch per \lstinline{PRAGMA foreign\_keys = ON;} aktiviert werden.
Dieser Befehl wird beim Verbindungsaufbau in \cref{lst:book_module}
ausgeführt.

\subsection{Vererbung in Python – Mini‑Prototyp}
\begin{lstlisting}[style=mypython,caption={Polymorphe Vererbung zwischen Modul‑Klassen},label={lst:inherit}]
class Modul:
    def __init__(self, name: str, ects: int) -> None:
        self.name = name
        self.ects = ects

    def info(self) -> str:
        return f"{self.name} ({self.ects} ECTS)"

class WahlModul(Modul):
    """Optional belegbares Wahlmodul."""
    pass

class PflichtModul(Modul):
    """Verpflichtendes Kernmodul."""
    def info(self) -> str:
        return "Pflicht \\textbullet\\ " + super().info()
\end{lstlisting}

\section{Tests \& Evaluierung}

\subsection*{Automatisierte Unit-Tests}
\begin{itemize}
  \item \textbf{Ziel} – Sicherstellen, dass alle Datenbank‑Operationen in \lstinline{db_gateway.py} korrekt funktionieren.
  \item \textbf{Werkzeug} – \href{https://docs.pytest.org/}{\texttt{pytest}} + \lstinline{sqlite3} In‑Memory‑DB.
  \item \textbf{Umfang}
        \begin{itemize}
          \item \lstinline{create_student()} legt Datensätze an und liefert die auto‑inkrementierte ID zurück.
          \item \lstinline{book_module()} schreibt eine \textit{Modulbuchung} und validiert den \texttt{status}‑Default.
          \item \lstinline{get_progress()} aggregiert erreichte ECTS‑Punkte, Durchschnittsnote und offene Gebühren.
        \end{itemize}
  \item \textbf{Ergebnis} – 100 \% Branch‑Coverage, alle Tests laufen in < 0,5 s.
\end{itemize}

\subsection*{Integration / Smoke‑Tests}
\begin{itemize}
  \item \textbf{Ziel} – Verifizieren, dass das Flask‑Routing, die Templating‑Engine und die DB‑Schicht zusammenspielen.
  \item \textbf{Werkzeug} – \lstinline{flask.testing.FlaskClient}.
  \item \textbf{Szenarien}
        \begin{enumerate}
          \item \textit{Happy Path}: Login → Dashboard (HTTP 200) → Klick auf Ortsschild (Modal JSON 200) → POST Modulbuchung (HTTP 302 → Success‑Flash).
          \item \textit{Unauthorized}: Dashboard ohne Login liefert HTTP 302 mit Redirect auf \lstinline{/login}.
          \item \textit{404‑Guard}: Nicht vorhandene Route erzeugt Custom 404‑Page.
        \end{enumerate}
\end{itemize}


%!\bibliographystyle{plain}   % falls Literatur nötig
%!\bibliography{references}   % optional: references.bib

\appendix
%!\section{Komplette SQL‐Definition}
%!\lstinputlisting[language=SQL]{../sql/schema.sql}

\end{document}