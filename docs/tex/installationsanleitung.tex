\documentclass[11pt,a4paper]{article}

% Pakete
\usepackage[utf8]{inputenc}
\usepackage[ngerman]{babel}
\usepackage[margin=2cm]{geometry}
\usepackage{helvet}
\renewcommand{\familydefault}{\sfdefault}
\usepackage{hyperref}
\usepackage{enumitem}
\usepackage{xcolor}

% Formalia gemäß Aufgabenstellung
\usepackage{setspace}
\onehalfspacing % 1,5-facher Zeilenabstand
\setlength{\parskip}{6pt} % 6pt Absatzabstand
\setlength{\itemsep}{0pt}

% WICHTIG: Silbentrennung aktivieren
\hyphenation{Installations-anleitung Studien-navigator Flask-Weban-wendung Auto-motive-Info-tainment-Meta-pher}
\sloppy % Erlaubt etwas flexiblere Zeilenumbrüche

% Hyperref Setup
\hypersetup{
    colorlinks=true,
    linkcolor=blue,
    urlcolor=blue,
    pdftitle={Installationsanleitung IU Studiennavigator},
    pdfauthor={Teresa Ignatzek}
}

\pagestyle{empty}

\usepackage{titlesec}
\titleformat{\section}{\fontsize{12}{14}\bfseries}{\thesection}{1em}{}
\titlespacing*{\section}{0pt}{6pt}{3pt}
\titleformat{\subsection}{\fontsize{12}{14}\bfseries}{\thesubsection}{1em}{}
\titlespacing*{\subsection}{0pt}{4pt}{2pt}

\begin{document}

% Header
\noindent
\textbf{\large Installationsanleitung: IU Studiennavigator}\\[0.2cm]
\textbf{Projekt:} Objektorientierte Programmierung mit Python (DLBDSOOFPP01\_D)\\
\textbf{Studentin:} Teresa Ignatzek | \textbf{Datum:} \today

\vspace{0.3cm}

% GitHub-Link kompakt
\noindent
\colorbox{lightgray}{\parbox{\textwidth}{
\textbf{GitHub:} \url{https://github.com/3vilQu33n/iu-studiennavigator}
}}

\vspace{0.3cm}

\section*{Systembeschreibung}
Der \textbf{IU Studiennavigator} ist eine Flask-Webanwendung zur Verwaltung des Studienfortschritts mit innovativer Automotive-Infotainment-Metapher und dynamischer SVG-Roadmap-Visualisierung. Plattformunabhängig (Windows, macOS, Linux).

\section*{Voraussetzungen}
Python 3.12+, pip, moderner Browser (Chrome/Firefox/Safari/Edge), Git (optional)

\section*{Installation}

\noindent\textbf{1. Repository herunterladen}
\begin{verbatim}
git clone https://github.com/3vilQu33n/iu-studiennavigator
cd iu-studiennavigator
\end{verbatim}

\noindent\textbf{2. Dependencies installieren}
\begin{verbatim}
pip install -r requirements.txt
\end{verbatim}
\textit{Hinweis:} Auf macOS/Linux ggf. \texttt{pip3} verwenden.

\noindent\textbf{3. Anwendung starten}
\begin{verbatim}
python app.py
\end{verbatim}
\textit{Hinweis:} Auf macOS/Linux ggf. \texttt{python3} verwenden.

\noindent\textbf{4. Browser öffnen:} \url{http://localhost:5050}

\section*{Demo-Login}
\textbf{E-Mail:} \texttt{demo.student@study.ignatzek.org} | \textbf{Passwort:} \texttt{DemoStudent\#2024}

\section*{Alternative: Docker-Installation}
\begin{verbatim}
# Image bauen
docker build -t iu-studiennavigator:latest .

# Container starten
docker run -d --name iu-studiennavigator \
  --restart unless-stopped -p 5050:5000 \
  -v $(pwd)/data:/app/data --env-file .env \
  iu-studiennavigator:latest
\end{verbatim}
Anwendung dann unter: \url{http://localhost:5050}\\
\textit{Hinweis:} Für Unraid siehe Deployment-Dokumentation im Repository.

\section*{Funktionsübersicht}
Dashboard (Automotive-Roadmap), Semester-Ansicht, Modulbuchung, Prüfungsanmeldung (Two-Stage Dropdown), Gebührenverwaltung

\section*{Tech Stack}
Flask 3.0.3, SQLite3, Python 3.12+, MVC mit Repository Pattern, Vanilla JavaScript/CSS/SVG, Docker

\section*{Fehlerbehebung}
\textbf{Port belegt:} In \texttt{app.py} Port ändern: \texttt{app.run(debug=True, port=5001)}\\
\textbf{Module fehlen:} \texttt{pip install --upgrade -r requirements.txt}\\
\textbf{Tests:} \texttt{pytest} (über 1000 Tests verfügbar)

\vspace{0.2cm}
\noindent
\textit{Kontakt: teresa@ignatzek.de}

\end{document}