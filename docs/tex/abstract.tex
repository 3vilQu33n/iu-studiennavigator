\documentclass[11pt,a4paper]{article}

% Pakete
\usepackage[utf8]{inputenc}
\usepackage[ngerman]{babel}
\usepackage[margin=2cm]{geometry}
\usepackage{helvet} % Arial-ähnliche Schriftart
\renewcommand{\familydefault}{\sfdefault}
\usepackage{parskip} % Absätze mit 6pt Abstand
\setlength{\parskip}{6pt}
\usepackage{setspace}
\onehalfspacing % 1,5-facher Zeilenabstand

% WICHTIG: Silbentrennung aktivieren
\hyphenation{Studien-fort-schritts-Ver-waltung Auto-motive-Info-tainment-Meta-pher Semester-Meilen-steinen objekt-orientierte Programmierung}
\sloppy % Erlaubt etwas flexiblere Zeilenumbrüche für bessere Silbentrennung

% Hyperref Setup
\usepackage{hyperref}
\hypersetup{
    colorlinks=true,
    linkcolor=black,
    urlcolor=blue,
    pdftitle={Abstract: IU Studiennavigator},
    pdfauthor={Teresa Ignatzek}
}

% Titel-Formatierung (12pt für Überschriften)
\usepackage{titlesec}
\titleformat{\section}{\fontsize{12}{14}\bfseries}{\thesection}{1em}{}
\titleformat{\subsection}{\normalsize\bfseries}{\thesubsection}{1em}{}

% Keine Seitenzahlen
\pagestyle{empty}

\begin{document}

% Header
\noindent
\textbf{\large Abstract: IU Studiennavigator}\\[0.2cm]
\textbf{Portfolio-Projekt:} Objektorientierte Programmierung mit Python\\
\textbf{Studentin:} Teresa Ignatzek\\
\textbf{Datum:} \today

\vspace{0.5cm}

\section*{Projektzusammenfassung}

Im Rahmen des Portfolio-Projekts wurde der \textbf{IU Studiennavigator} entwickelt – eine webbasierte Studienfortschritts-Verwaltung mit innovativer Automotive-Infotainment-Metapher. Die Anwendung visualisiert den akademischen Fortschritt mittels eines interaktiven Fahrzeugs, das sich entlang einer SVG-Roadmap zwischen Semester-Meilensteinen bewegt.

Das System demonstriert fortgeschrittene objektorientierte Programmierung in Python mit strikter Anwendung der OOP-Prinzipien (Komposition, Aggregation, Vererbung, Polymorphie, Encapsulation) und implementiert eine saubere MVC-Architektur mit Repository Pattern.

\section*{Herangehensweise}

Das Projekt wurde in drei strukturierten Phasen entwickelt:

\textbf{Phase 1 – Konzeption:} Analyse der Anforderungen, Erstellung eines UML-Klassendiagramms mit Entity-Klassen, kritische Diskussion der Modellierungsentscheidungen (Komposition vs. Aggregation), Machbarkeitsanalyse für die Python-Implementierung.

\textbf{Phase 2 – Reflexion \& Entwurf:} Untersuchung der Umsetzung objektorientierter Konzepte in Python, Erstellung der Gesamtarchitektur mit zusätzlichen Controller-, Repository- und Service-Klassen, Erweiterung des UML-Diagramms zur vollständigen Systemarchitektur.

\textbf{Phase 3 – Implementierung:} Entwicklung des funktionsfähigen Prototyps mit Flask-Backend, SQLite-Datenbank, umfangreicher Test-Suite (1000+ Tests), und innovativem SVG-basiertem Frontend mit dynamischer Fahrzeugpositionierung entlang tatsächlicher Pfadkoordinaten.

\section*{Erfolge und Herausforderungen}

\subsection*{Was lief besonders gut}

Die strikte Anwendung objektorientierter Prinzipien führte zu einer wartbaren, erweiterbaren Codebasis. Die Automotive-Metapher erwies sich als intuitive und kreative Visualisierung des Studienfortschritts. Die SVG-basierte dynamische Positionierung ermöglicht responsive Darstellung ohne hardcodierte Pixel-Werte. Das Repository Pattern mit DBGateway sorgt für klare Trennung von Business-Logik und Datenzugriff.

Die umfangreiche Test-Suite (über 1000 Tests) mit hoher Coverage gewährleistet Codequalität und erleichtert Refactoring. Die Implementierung von Komposition (Student ◆→ Login), Aggregation (Student ◇→ Einschreibung) und Vererbung (Pruefungsleistung extends Modulbuchung) demonstriert tiefes Verständnis der OOP-Beziehungen.

\subsection*{Herausforderungen}

Die größte technische Herausforderung war die präzise Berechnung der Fahrzeugposition entlang komplexer SVG-Pfade. Die Koordinatentransformation zwischen Haupt-Dashboard und Popup-Miniaturansicht erforderte sorgfältiges Debugging. UTF-8-Encoding-Probleme mit deutschen Umlauten mussten systematisch über alle Python-Dateien hinweg behoben werden.

Die Balance zwischen akademischen Anforderungen (strikte private Methoden mit \texttt{\_\_double\_underscore}) und praktischer Wartbarkeit erforderte sorgfältige Abwägung. Die Migration der Datenbankstruktur von impliziten zu expliziten Many-to-Many-Beziehungen für das Prüfungsarten-System war komplex, verbesserte aber die Flexibilität erheblich.

\section*{Besondere Aspekte}

Besonders stolz bin ich auf die innovative Visualisierung: Die Automotive-Metapher verbindet Gaming-Elemente mit akademischer Datenverwaltung und macht Studienfortschritt intuitiv erfassbar. Die Verwendung echter SVG-Pfadkoordinaten statt hardcodierter Positionen ermöglicht echte Responsiveness.

Die strikte Architektur mit minimalen public Interfaces und umfangreichen private Methoden demonstriert professionelles Software-Design. Die Dual-Repository-Strategie (akademische Version mit realen Daten, Portfolio-Version mit Demo-Daten) zeigt durchdachtes Projektmanagement.

Die Docker-Integration ermöglicht unkompliziertes Deployment und demonstriert DevOps-Kenntnisse über reine Programmierung hinaus.

\section*{Erkenntnisse und Ausblick}

Das Projekt verdeutlichte die praktische Bedeutung objektorientierter Prinzipien: Komposition und Aggregation sind mehr als theoretische Konzepte – sie bestimmen Lifecycle-Management und Systemarchitektur fundamental. Test-Driven Development mit hoher Coverage ist unverzichtbar für robuste Anwendungen.

Die Ergebnisse sind vielfältig verwendbar: als Portfolio-Projekt für Bewerbungen, als Basis für erweiterte Features (Notenspiegel, Modulempfehlungen, Studienplan-Generator), und als Demonstration moderner Python-Webentwicklung. Eine Containerisierung mit Docker wurde bereits implementiert und ermöglicht flexibles Deployment.

\vspace{0.3cm}
\noindent
\textit{Dieses Projekt demonstriert, dass objektorientierte Programmierung nicht nur theoretische Eleganz, sondern praktischen Mehrwert für wartbare, erweiterbare Softwaresysteme bietet.}

\end{document}